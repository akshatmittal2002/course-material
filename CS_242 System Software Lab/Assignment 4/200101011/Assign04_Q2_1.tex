\documentclass[12pt]{article}
%Including packages to be used.
\usepackage{amsmath}
\usepackage[left=3.5cm,right=3.5cm,top=2cm,bottom=2cm]{geometry}
\usepackage{color}
\usepackage[utf8]{inputenc}
%Make title with Title , author and date
\title{Hello World!}
\author{Akshat Mittal}
\date{November 17, 2021}
%Begin document
\begin{document}
%Remove page number
\pagenumbering{gobble}
%Print title
\maketitle
%Start section 1
\section{Getting Started}

\textbf{Hello World!} Today I am learning \LaTeX{}. \LaTeX{} is a great program for writing math. I can write in line math such as $a^2+b^2=c^2$. I can also give equations
their own space:
%Print equation 1.
\begin{equation}
\gamma^2+\theta^2=\omega^2 \label{eq1}
\end{equation}
%Flush left space. If we donot do this, our text will be indented slightly.
\begin{flushleft}
``Maxwell's equations" are named for James Clark Maxwell and are as follow:
%Write 4 equaltions of Maxwell, aligned with '=' and text; give equation numbers.
\begin{align}
\vec{\nabla}\cdot\vec{E}\quad&=\quad\frac{\rho}{\epsilon_0} &&\text{Gauss's Law} \label{eq2}\\
\vec{\nabla}\cdot\vec{B}\quad&=\quad 0 &&\text{Gauss's Law for Magnetism} \label{eq3}\\
\vec{\nabla}\times\vec{E}\quad&=\quad -\frac{\partial \vec{B}}{\partial{t}} &&\text{Faraday's Law of Induction}\label{eq4}\\
\vec{\nabla}\times\vec{B}\quad&=\quad\mu_0\left(\epsilon_0\frac{\partial \vec{E}}{\partial t} + \vec{J}\right) &&\text{Ampere's Circuital Law}\label{eq5}
\end{align}

Equations $\color{blue}{2}$, $\color{blue}{3}$, $\color{blue}{4}$, and $\color{blue}{5}$ are some of the most important in Physics.

%Start section 2.
\section{What about Matrix Equations?}

%Print matrix equation.
\[
\begin{pmatrix}
{a_{11}} & {a_{12}} & \dots & {a_{1n}} \\
{a_{21}} & {a_{22}} & \dots & {a_{2n}} \\
 \vdots &\vdots & \ddots &\vdots\\
{a_{n1}} & {a_{n2}} & \dots & {a_{nn}}
\end{pmatrix}
\begin{bmatrix}
{v_1}\\
{v_2}\\
 \vdots\\
{v_n}
\end{bmatrix}
=
\begin{matrix}
{w_1}\\
{w_2}\\
 \vdots\\
{w_n}
\end{matrix}
\]

%End flush and end document.
\end{flushleft}
\end{document}
